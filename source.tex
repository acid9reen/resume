\documentclass[12pt]{article}
\usepackage[russian,english]{babel}
\usepackage{cmbright}
\usepackage{enumitem}
\usepackage{fancyhdr}
\usepackage{fontawesome5}
\usepackage{geometry}
\usepackage{hyperref}
\usepackage[sf]{libertine}
\usepackage{microtype}
\usepackage{paracol}
\usepackage{supertabular}
\usepackage{titlesec}
\usepackage{fontspec}
\hypersetup{colorlinks, urlcolor=black, linkcolor=black}

% Geometry
\geometry{hmargin=1.75cm, vmargin=2.5cm}
\columnratio{0.65, 0.35}
\setlength\columnsep{0.05\textwidth}
\setlength\parindent{0pt}
\setlength{\smallskipamount}{8pt plus 3pt minus 3pt}
\setlength{\medskipamount}{16pt plus 6pt minus 6pt}
\setlength{\bigskipamount}{24pt plus 8pt minus 8pt}

% Style
\pagestyle{empty}
\titleformat{\section}{\scshape\LARGE\raggedright}{}{0em}{}[\titlerule]
\titlespacing{\section}{0pt}{\bigskipamount}{\smallskipamount}
\newcommand{\heading}[2]{\centering{\sffamily\Huge #1}\\\smallskip{\large{#2}}}
\newcommand{\entry}[4]{{{\textbf{#1}}} \hfill #3 \\ #2 \hfill #4}
\newcommand{\tableentry}[3]{\textsc{#1} & #2\expandafter\ifstrequal\expandafter{#3}{}{\\}{\\[6pt]}}

\begin{document}
% Это для того, чтобы текст не вылазил за край параграфа
\emergencystretch 3em

\vspace*{\fill}

\begin{paracol}{2}

% Name & headline
\heading{Руслан Смирнов}{Разработчик Python}

\switchcolumn

% Identity card
\vspace{0.01\textheight}
\begin{supertabular}{ll}
  \footnotesize\faPhone & +7 930 671 09 95 \\
  \footnotesize\faEnvelope & \href{mailto:smirnov\_ruslan@outlook.com}{smirnov\_ruslan@outlook.com} \\
  \footnotesize\faTelegram & \href{https://t.me/acid9reen}{@acid9reen} \\
  \footnotesize\faGithub & \href{https://github.com/acid9reen}{acid9reen} \\
\end{supertabular}

\smallskip
\switchcolumn*

\section{Высшее образование}

\entry{ННГУ им. Лобачевского}{Магистр}{Нижний Новгород}{2022 -- 2024}
\begin{itemize}[noitemsep,leftmargin=3.5mm,rightmargin=0mm,topsep=6pt]
  \item Направление: Прикладная Математика и Информатика
  \item Кафедра: Алгебры, Геометрии и Дискретной Математики
\end{itemize}

\medskip

\entry{ННГУ им. Лобачевского}{Бакалавр}{Нижний Новгород}{2018 -- 2022}
\begin{itemize}[noitemsep,leftmargin=3.5mm,rightmargin=0mm,topsep=6pt]
    \item Направление: Компьютерные Науки и Приложения
    \item Кафедра: Алгебры, Геометрии и Дискретной Математики
\end{itemize}

\switchcolumn

\section{Навыки}
\begin{supertabular}{rl}
  \tableentry{\footnotesize\faCode}{Python \textperiodcentered{} FastAPI \textperiodcentered{} PyTorch}{}
  \tableentry{}{SQLAlchemy \textperiodcentered{} PostgreSQL}{}
  \tableentry{}{Docker \textperiodcentered{} Docker Compose}{}
  \tableentry{}{Git \textperiodcentered{} Linux \textperiodcentered{} OpenCV}{}
  \tableentry{}{}{}

  \tableentry{\footnotesize\faLanguage}{English --- Intermediate}{}
\end{supertabular}

\switchcolumn*

\section{Опыт работы}

\entry{ННГУ им. Лобачевского}{Лаборант-исследователь}{Нижний Новгород}{Октябрь 2022 -- Декабрь 2023}
\begin{itemize}[noitemsep,leftmargin=3.5mm,rightmargin=0mm,topsep=6pt]
  \item Обязанности: Разработка ПО для автоматической разметки электрограммы с
    использованием сверточных нейронных сетей
  \item Стек: Python3.10+, PyTorch, ONNX, MLFlow
\end{itemize}

\medskip

\entry{АО Гринатом}{Python разработчик}{Нижний Новгород}{Апрель 2022 -- Н.В.}
\begin{itemize}[noitemsep,leftmargin=3.5mm,rightmargin=0mm,topsep=6pt]
  \item Обязанности: Написание RESTful веб-сервисов. Поиск, сравнение,
    реализация, обучение моделей машинного обучения для задач биометрии. Модели
    детекции лица (mtcnn, yoloface, retinaface), оценки качества (sdd-fiqa,
    faceqnet)
  \item Стек: Python3.10+,  FastAPI, Docker, Docker Compose, PostgreSQL,
    SQLAlchemy, PyTorch, Redis, Celery, ONNX, WandB, OpenCV
\end{itemize}

\switchcolumn


\section{Курсы}
\begin{supertabular}{rl}
  \tableentry{2021}{\textbf{Sber corporate university}}{}
  \tableentry{}{Python, Machine Learning}{spaceafter}
  \tableentry{2020}{\textbf{Artezio academy 3}}{}
  \tableentry{}{Python}{spaceafter}
  \tableentry{2019}{\textbf{Intel Delta 11 Course}}{}
  \tableentry{}{Python, C++}{spaceafter}
\end{supertabular}

\end{paracol}

\vspace*{\fill}

\end{document}
